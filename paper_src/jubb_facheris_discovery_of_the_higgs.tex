\documentclass{sig-alternate-10pt}

\newcommand{\ttt}{\texttt}



\title{Discovery of the Higgs Boson\\ \Large A Social Analysis}
\author{
	Chae Jubb and Patrick Facheris \\
	\ttt{\{ecj2122,plf2110\}@columbia.edu}
}
\date{23 December 2014}

\begin{document}
\maketitle

\section{Problem Statement}
%Par. 1 general motivation, why is that pb area important
Finding reliable information on the internet is important.
This is especially important during times of crisis and extreme innovation.
In its microblogging form, Twitter provides quick access to information that anyone can post.
While news stories are breaking, it is important to distribute reliable information.
We can approximate this pursuit by finding the key influencers on this platform
For various reasons, certain individuals become a rallying point or a key participant in the spread of information.
Because a few people will dominate the information sphere, identifying these people is important.
Having this information, the media and press might target this group.
Doing this is obviously much more effective and efficient than contacting those without influence and power.

%Par. 2~3 narrow down: specify what choices you make and important assumptions (without being too technical)
% Hubs and authorities; SVM
[Paragraph 2 here: more general maybe?]

We focus on three auxiliary data sets: retweets, replies, and mentions.
These data sets serve three distinct purposes.
\begin{description}
\item [Retweets]
    This interaction is the most passive of those listed here.
    It often serves simply as an acknowledgement of another's communication.
    We assume this to be an indicator of a very weak relationship.
\item [Replies]
    Of the three, replies indicate the most direct and personal form of interaction between two users.
    It is a direct reach to initiate (or continue) a conversation with another user.
    We assume this to be an indicator of a tight relationship, especially if the communication is two-way.
\item [Mentions]
    Mentions are the least straightforward because their use ranges the most widely.
    This range is mostly in part due to the wide range of a ``shout-out''.
    It could be an inviation for communication or simply equivalent to the CC line on an email.
\end{description}
Through appropriate analysis and synthesis, we attempt to reconstruct the social graph using these auxiliary data sets.

%Par. 4 first sentence should describe your problem formulation and contribution. Typically people start this paragraph with “In this paper, we ...”
In this paper, we attempt to reconstruct a directed social network using auxiliary data concerning interactions on that network.
We focus on the spread of information surrounding the time period of the announcement of the discovery of a particle with properties similar to those of the theoretical Higgs Boson.
[FINISH PARAGRAPH 4]

\section{Related Work}


\section{Results}


\end{document}
